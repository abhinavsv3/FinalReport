% University Assignment Title Page 
% LaTeX Template
% Version 1.0 (27/12/12)
%
% This template has been downloaded from:
% http://www.LaTeXTemplates.com
%
% Original author:
% WikiBooks (http://en.wikibooks.org/wiki/LaTeX/Title_Creation)
%
% License:
% CC BY-NC-SA 3.0 (http://creativecommons.org/licenses/by-nc-sa/3.0/)
% 
% Instructions for using this template:
% This title page is capable of being compiled as is. This is not useful for 
% including it in another document. To do this, you have two options: 
%
% 1) Copy/paste everything between \begin{document} and \end{document} 
% starting at \begin{titlepage} and paste this into another LaTeX file where you 
% want your title page.
% OR
% 2) Remove everything outside the \begin{titlepage} and \end{titlepage} and 
% move this file to the same directory as the LaTeX file you wish to add it to. 
% Then add \input{./title_page_1.tex} to your LaTeX file where you want your
% title page.
%
%%%%%%%%%%%%%%%%%%%%%%%%%%%%%%%%%%%%%%%%%
%\title{Title page with logo}
%----------------------------------------------------------------------------------------
%	PACKAGES AND OTHER DOCUMENT CONFIGURATIONS
%----------------------------------------------------------------------------------------

\documentclass[12pt]{article}
\usepackage[english]{babel}
\usepackage[utf8x]{inputenc}
\usepackage{amsmath}
\usepackage{graphicx}
\usepackage{afterpage}
\usepackage{eurosym}
\usepackage[colorinlistoftodos]{todonotes}

\newenvironment{dedication}
  {
\clearpage           % we want a new page
   \thispagestyle{empty}% no header and footer
   \vspace*{\stretch{1}}% some space at the top 
   \itshape             % the text is in italics
   \raggedleft          % flush to the right margin
  }
  {\par % end the paragraph
   \vspace{\stretch{3}} % space at bottom is three times that at the top
   \clearpage           % finish off the page
  }

\begin{document}

\begin{titlepage}

\newcommand{\HRule}{\rule{\linewidth}{0.5mm}} % Defines a new command for the horizontal lines, change thickness here

\center % Center everything on the page
 
%----------------------------------------------------------------------------------------
%	HEADING SECTIONS
%----------------------------------------------------------------------------------------

\textsc{\LARGE Universitat Politècnica de Catalunya}\\[1.5cm] % Name of your university/college
\textsc{\Large Bachelors in Computer Science and Engineering}\\[0.5cm] % Major heading such as course name
\textsc{\large Minor Heading}\\[0.5cm] % Minor heading such as course title

%----------------------------------------------------------------------------------------
%	TITLE SECTION
%----------------------------------------------------------------------------------------

\HRule \\[0.4cm]
{ \huge \bfseries Graph and matrix algorithms for visualizing high dimensional data}\\[0.4cm] % Title of your document
\HRule \\[1.5cm]
 
%----------------------------------------------------------------------------------------
%	AUTHOR SECTION
%----------------------------------------------------------------------------------------

\begin{minipage}{0.4\textwidth}
\begin{flushleft} \large
\emph{Director:}\\
Dr. Ricard Gavalda Mestre 
\end{flushleft}
\end{minipage}
~
\begin{minipage}{0.4\textwidth}
\begin{flushright} \large
\emph{Co-Director:} \\
Dr. Marta Arias Vincente % Supervisor's Name
\end{flushright}
\end{minipage}\\[2cm]

\begin{minipage}{0.4\textwidth}
\begin{flushleft} \large
\emph{Bachelors Thesis of :}\\
Abhinav Shankaranarayanan Venkataraman
\end{flushleft}
\end{minipage}

% If you don't want a supervisor, uncomment the two lines below and remove the section above
%\Large \emph{Author:}\\
%John \textsc{Smith}\\[3cm] % Your name

%----------------------------------------------------------------------------------------
%	DATE SECTION
%----------------------------------------------------------------------------------------
\bigskip
\bigskip
{\large June 27,2016}\\[2cm] % Date, change the \today to a set date if you want to be precise

%----------------------------------------------------------------------------------------
%	LOGO SECTION
%----------------------------------------------------------------------------------------

\includegraphics{logo.png}\\[1cm] % Include a department/university logo - this will require the graphicx package
 
%----------------------------------------------------------------------------------------

\vfill % Fill the rest of the page with whitespace

\end{titlepage}

\newpage




\begin{dedication}
To my Mother, Father, Professors and Friends. I owe a lot to My professors Ricard Gavalda and Marta Arias and to Babaji at Gurudwara
\end{dedication}
\newpage
\clearpage
\newpage
\newpage

\begin{abstract}
Motivated by the problem of understanding data from
the medical domain, we consider algorithms for visually representing 
highly dimensional data so that "similar" entities appear close together. We will study, 
implement and compare several algorithms based on graph and on matrix
representation of the data. The first kind are known as "community detection"
algorithms, the second kind as "clustering" algorithms. The implementations
should be robust, scalable, and provide a visually appealing representation
of the main structures in the data.

\end{abstract}

\newpage
\clearpage
\newpage
\section*{Acknowledgement}
I would like to Acknowledge the support provided by my faculty and admins at my home university -- SASTRA University, Thanjavur and UPC Barcelona for supporting me throughout the project.

\newpage

\tableofcontents
\newpage


\section{Introduction}
\par In this section	we provide an overview of the entire work. We mention the context of the project we have studied, approaches that we have used, goal of the project. We also provide the intended planning, economic estimate and sustainability of the work that has been done.


\subsection{Context Of the Project}
\par In the present day scenario, the modern science of algorithms and graph theory has brought significant advances to our understanding of complex data. Many complex systems are representable in the form of graphs. Graphs have time and again been used to represent real world networks. One of the most pertinent feature of graphs representing real system is community structures or otherwise known as clusters. Community can be defined as the organization of vertices in groups or clusters, with many edges joining the vertices of the same cluster and comparatively fewer vertices joining the vertices in another neighbouring cluster. Such communities form an independent compartment of a graph exhibiting similar role.
Thus, Community detection is the key for understanding the structure of complex graphs, and ultimately educe information from them.

\subsection{Approaches}
\subsection{For Community Identification}
Virtually in every scientific field dealing with empirical data, primary approach to get a first impression on the data is by trying to identify groups having "similar" behaviour in data. There are numerous methods to achieve this objective of which 

\begin{itemize}
\item Community Detection
\item Clustering
\end{itemize}

\subsubsection{Community Detection}
\paragraph{Definition}
Communities are a part of the graph that has fewer ties with the rest of the system. Commmunity detection traditionally focueses on the graph structures while clusting algorithms focuses on node attributes. 


\subsubsection{Clustering}

Traditional Clustering Methods are as follows:
\begin{itemize}
\item Graph Partitioning 
\item Hierarchical Clustering
\item Partitional Clustering
\item Spectral Clustering

\end{itemize}


\subsubsection{For  Visualization}

\subsubsection{Computational Complexity}
 The estimate of the amount of resources required for by the algorithm to perform a task is defined as computational complexity. The humongous amount of data on the real graphs or real networks that are available in the current scenario causes the efficiency of the clustering algorithm to be crucial.


community finding and clustering. separately,
    visualization tools
\subsection{Goal of the Project}

\subsection{Planning}
\subsubsection{Task Description}
The tasks for the project have been subdivided into various task phases which are enumerated below : 
\begin{itemize}
\item \textbf{Required knowledge acquisition}\\
Before any immersion into the real topic, it was necessary to acquire the knowledge
necessary to understand the problem. In this phase we familiarize with the term
modularity, Louvain algorithm for community detection and various other algorithms
used for community detection.
Acquisition of knowledge about visualization tools to be used and make conversant with python is also
required.
\item \textbf{Paper Analysis}\\ In this phase we analyze and compare several works about
community detection and clustering algorithm over high dimensional graph-like data.
Doing this we became conscious of functionalities that our proposal should have and we
are thus able to guide all the subsequent phases.

\item \textbf{Design and Implementation} \\ In this phase the project is designed and coded implementing all the functionalities of the solution. 
\item \textbf{Testing I}\\
In this phase we test the program in order to identify errors in
the implementation. It includes the successive recoding.
\item \textbf{Testing II}\\
In this phase we perform tests over synthetic and real data
streams. We evaluate the performance of the program and we study
the effects of concept drift.
\item \textbf{Report Writing}\\
In this phase the report of the project is written.
\end{itemize} 
\subsection{Economic Budget}
\subsubsection{An Introduction to Economic Budget}
Economic management is primarily based on an estimate of income and expenditure called as
budget. Development of a sustainable budget leads to proper economic management of the
project. Budget and sustainability is one of the most important phase of the project
management. In this phase we analyze the budget for the project. We also aim at providing an
estimate of the project budget and optimize the same. We look at the expenditure from various
aspects such as software costs, hardware costs, license costs and human resource costs.
Additionally we also account the software for its sustainability. One important factor to note is
that the budget that we describe in this section is subject to change and it may increase
depending on the unexpected obstacles that we may face. For an instance when we don’t get
the expected results with a particular software we may have to go in for another software that
may incur extra installation and operational charges.
\subsubsection{Estimation of Economic Budget}
We divide the overall expenditure into three categories namely hardware, software and human
resources. One very important factor that we need to consider is that we only get an estimate
of the total cost. This may vary depending on the systems in use. To calculate the amortization
we consider to factors namely, first the overall life of the hardware or software in use. Second
that the project is completed in 5 months. Hence the amortization cost comes one eighth
of the actual life of the component.

\paragraph{Hardware Budget} Hardware budget accounts for the actual and the amortized costs of the hardware elements
used by the project. The cost is fictitious as it has not been developed commercially. Table 1
intents to estimate the economic cost of each of the hardware component of the project.
\\\\
\begin{tabular}{|p{1cm}||p{3cm}|p{2cm}|p{3cm}|p{3cm}|}
 \hline
 \multicolumn{5}{|c|}{Table 1 - Hardware Budget} \\
 \hline
 Sno: & Hardware Component&Useful Life &Total Cost(in \euro) &Amortized Cost(in \euro)\\
 \hline
1   & PC System  &4 &  1000\euro  & 125 \euro \\
\hline
\hline
   & \textbf{Total}  &  &  \textbf{1000}\euro  & \textbf{125} \euro \\
 \hline
\end{tabular}

\paragraph{Software Budget}
The software budget shows an estimate for the various software used in the project along with
the estimate of the software costs. It is a myth that the software doesn’t get old with time just
as a software gets but it wears out with time. Thus for every software there is a fixed time
during which it gives maximum performance. In addition freeware software and open source
software incur no cost. The cost is fictitious as it has not been developed commercially. Table 2
intents to estimate the economic cost of each of the software component of the project.
\\ \\
\begin{tabular}{|p{1cm}||p{3cm}|p{2cm}|p{3cm}|p{3cm}|}
 \hline
 \multicolumn{5}{|c|}{Table 2 - Software Budget} \\
 \hline
 Sno: & Software Component&Useful Life &Total Cost(in \euro) &Amortized Cost(in \euro)\\
 \hline
1   & Linux OS  &5 &  0\euro  & 0 \euro \\
2   & JavaScript Engine  &1 &  0\euro  & 0 \euro \\
3   & Python Components  &1 &  0\euro  & 0 \euro \\
4   & Web.py  &1 &  0\euro  & 0 \euro \\
5   & TexMaker  &1 &  0\euro  & 0 \euro \\

\hline
\hline
   & \textbf{Total}  &  &  \textbf{0}\euro  & \textbf{0} \euro \\
 \hline
\end{tabular}
\\ \\ 
\paragraph{Human Resource Budget}
The human resource budget deals with the overall expenditure spent on human resources.
Every phase of the project has a cost associated with it in per hour calculation.
The cost is fictitious as it has not been developed commercially. Table 3 intents to estimate the
economic cost of each of the phases of the project. The cost per hour is intended as an
approximation of the current cost per work hour of young analysts and developers in our
environment.\\

\begin{tabular}{|p{0.5cm}||p{4cm}|p{2.5cm}|p{1cm}|p{1.5cm}|p{2cm}|}
 \hline
 \multicolumn{6}{|c|}{Table 3 - Human Resource Budget} \\
 \hline
 Sno: & Phase&Deadline &Hours &Cost(per hour in \euro)&Total(in \euro)\\
 \hline
1   & Required Knowledge Acqusition  &1 Mar 2016 &  70  & 15\euro/h & 1050 \euros \\
2   & Paper Analysis  &1 Apr 2016& 150 &  15\euro/h  & 2250 \euro \\
3   & Design and Implementation&30 Apr 2016 &230&  20\euro/h  & 4600 \euro \\
4   & Testing I  &15 May 2016&75 &15\euro/h  & 0 1125\euro \\
5   & Testing II  &31 May 2016&75 &  15\euro/h  & 0 1125\euro \\
5   & Report Writing  &15 Jun 2016&100 &  15\euro/h  & 1500\euro \\
\hline
\hline
   & \textbf{Total}  &  & 600&   & \textbf{10525} \euro \\
 \hline
\end{tabular}



\subsection{Sustainability}


\section{Background Knowledge}
In this section we present the background knowledge required to understand and solve the problem
\subsection{Graph Notation}
Graph ,G , is construct consisting of two finite sets, the set V = \{ $v_1,v_2, \ldots ,v_n$ \} of vertices and the set E = \{ $e_1,e_2, \ldots,e_n$  \} of edges where each edge is a pair of vertices from V, for instance,
\begin{center}
$e_i = (v_j,v_k)$
\end{center}
is an edge from $v_j$ to $v_k$ represented as G=(V,E).

\subsection{Matrix Notation}
\subsection{Equivalence between Graph and Matrix Represenation}
\subsection{State-of-the-art in Community finding}
\subsection{State-of-the-art in Clustering}
\subsection{State-of-the-art in Graph Visualization}

explain technical concepts in more detail. for exampleequivalence
     of graph and matrix representations.
     state-of-the art in community finding, and clustering
     state-of-the art in graph visualization

\section{Community Finding Algorithm}
\subsection{Louvain Algorithm}
\subsubsection{Introduction}
\subsubsection{Reasoning}
\subsubsection{Description}
\subsubsection{Implementation}
\subsubsection{Experiments}
\subsubsection{Result}

\section{Matrix Based Algorithm}
\subsection{Matrix Algorithm}
\subsubsection{Introduction}
\subsubsection{Reasoning}
\subsubsection{Description}
\subsubsection{Implementation}
\subsubsection{Experiments}
\subsubsection{Result}

\section{Visualization}
\subsection{Alchemy.js}
\subsubsection{Introduction}
\subsubsection{Reasoning}
\subsubsection{Description}
\subsubsection{Methods and Library}
\subsubsection{Result}

\section{Overall System Description}
\subsection{Alchemy.js}
\subsubsection{Introduction}
\subsubsection{Implementation Benefits}
\subsubsection{Description}
\subsubsection{Result}

\section{Conclusion}
\subsection{Goals Achieved}
\subsection{Revision of Planning and Budget}
\subsection{Future Works}
\subsection{Personal Conclusion}


\bibliography{Reference}
\end{document}



